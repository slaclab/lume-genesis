\documentclass{article}
\usepackage[utf8]{inputenc}

\title{Conversion of radiation fields from genesis2, genesis4, and SRW to electric fields in SI units}
\author{A.~Halavanau and C.~Mayes}
\date{July 2020}
%\usepackage{draftwatermark}
%\usepackage{natbib}
%\usepackage{graphicx}
\usepackage{amsmath}

\begin{document}

\maketitle



\section{Genesis Field to Electric Field}
The description of FEL radiation field format in the {\sc genesis} manual is scarce. The {\sc dfl} / {\sc rad.h5} file stores a complex array of \texttt{ncar} $\times$ \texttt{ncar} points for each simulation slice, over a transverse domain size $[-\texttt{dgrid}$, -$\texttt{dgrid}]$ m in both the horizontal and vertical dimensions.

Let us denote the radiation field value taken from {\sc dfl} file for $n$-th slice at $i,j$ transverse location as $F_{n,i,j}$.
The manual hints that integrated value of $|F_{n,i,j}|^2$ over the spatial domain yields the total FEL power of $n$-th slice.
Indeed, the value of 
\begin{equation}
P_n = \sum_{i,j}{|F_{n,i,j}|^2},
\end{equation}
 matches the value of integrated power, in units of Watt, reported in the output file for $n$-th slice. The values of $F_{n,i,j}$ returned by {\sc genesis} are therefore in the units of $\sqrt{\textrm{W}}$. 
 One can use this fact to devise the conversion factor for {\sc dfl} / {\sc rad.h5} to V/m.
We first recall that time-averaged energy density is given by (Jackson, Eq. 7.14):
\begin{equation}
    u = \frac{\epsilon_0}{2} |E|^2,
\end{equation}
where $\epsilon_0$ is vacuum permittivity. The local
radiation intensity (energy per unit area per unit time) is related to the electric field as:
\begin{equation}
    I = c u = c \frac{\epsilon_0}{2} |E|^2 = \frac{E^2}{2 Z_0},
\end{equation}
where $Z_0 = \sqrt{\mu_0/\epsilon_0} =  1/(\epsilon_0 c)$ is the impedance of free space. In SI units, $Z_0 = \pi\cdot119.9169832  ~\textrm{V}^2/\textrm{W}$ exactly.  The total power (energy per unit time) over an area $A$ can be approximated by
\begin{align*}
    P &= \int I dA \\
      &\approx \sum_{i,j}  \frac{E_{i,j}^2}{2 Z_0} \Delta^2
\end{align*}
where $E_{i,j}$ is the electric field value on an equally spaced grid with grid spacing $\Delta$.

According to the {\sc genesis} manual, $\Delta = 2 * \texttt{dgrid} / (\texttt{ncar}-1)$ m, and therefore we can express the electric field for $n$-th slice for a $i,j$-th point on the grid in V/m as:

\begin{equation}
 E_{n,i,j} =  F_{n,i,j} \sqrt{2 Z_0}/\Delta
\end{equation}

\section{SRW Wavefront to Electric Field}

SRW's radiation wavefront class \texttt{SRWWFr} allows several unit systems. The only documentation is in the code of \texttt{srwlib.py}:
\begin{verbatim}
    unitElFld = 1 #electric field units:
    0- arbitrary, 1- sqrt(Phot/s/0.1%bw/mm^2)
    2- sqrt(J/eV/mm^2) or sqrt(W/mm^2), depending on representation (freq. or time)
\end{verbatim}

Assuming the time representation, the conversion from an SRW field $F_\texttt{SRW}$ (in units of $\sqrt{\textrm{W}}/\textrm{mm}$) to electric field $E$ is therefore:
\begin{equation}
    E = F_\texttt{SRW} *  \sqrt{2 Z_0} * \frac{1000\textrm{mm}}{\textrm{m}}
\end{equation}




J.D.~Jackson, \emph{Classical Electrodynamics, Third Edition} (1999)

\end{document}
